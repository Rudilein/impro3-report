% -------- Reference by number (author) ----------------------------------
% %%% Doc: ftp://tug.ctan.org/pub/tex-archive/macros/latex/contrib/natbib/natbib.pdf
\usepackage[%
	%round,	%(default) for round parentheses;
	square,	% for square brackets;
	%curly,	% for curly braces;
	%angle,	% for angle brackets;
	%colon,	% (default) to separate multiple citations with colons;
	comma,	% to use commas as separaters;
	%authoryear,% (default) for author-year citations;
	numbers,	% for numerical citations;
	%super,	% for superscripted numerical citations, as in Nature;
	sort,		% orders multiple citations into the sequence in which they appear in the list of references;
%	sort&compress,    % as sort but in addition multiple numerical citations
%                   % are compressed if possible (as 3-6, 15);
	%longnamesfirst,  % makes the first citation of any reference the equivalent of
%                   % the starred variant (full author list) and subsequent citations
%                   %normal (abbreviated list);
	%sectionbib,      % redefines \thebibliography to issue \section* instead of \chapter*;
%                   % valid only for classes with a \chapter command;
%                   % to be used with the chapterbib package;
	%nonamebreak,     % keeps all the authors names in a citation on one line;
%                   %causes overfull hboxes but helps with some hyperref problems.
]{natbib}

%\RequirePackage[fixlanguage]{babelbib}
%\bibliographystyle{babalpha-fl}%

%%% Bibliography styles with natbib support
%\bibliographystyle{plainnat} % Numeric Labels, alphabatical order
%\bibliographystyle{abbrvnat} % same as plain, but shorter names
%\bibliographystyle{unsrtnat} % same as plain, but appeariance in order of citation
\bibliographystyle{alphaurl}    % labels are formed by author and year

%%% Bibliography styles according to DIN
%%% get from: http://www.ctan.org/tex-archive/biblio/bibtex/contrib/german/din1505/
%\bibliographystyle{alphadin}
%\bibliographystyle{abbrvdin}
%\bibliographystyle{plaindin}
%\bibliographystyle{unsrtdin}
%\bibliographystyle{bib/bst/alphadin-mod} % Modifiziert: Kleinere Abstaende vor ";" und kein "+" bei etal.

%%% Bibliography styles created with custombib
%%% Doc: ftp://tug.ctan.org/pub/tex-archive/macros/latex/contrib/custom-bib/makebst.pdf
%\bibliographystyle{bib/bst/AlphaDINFirstName}

% other BibTeX styles: http://www.cs.stir.ac.uk/~kjt/software/latex/showbst.html

% Listings

\lstset{
         basicstyle=\small\ttfamily, % Standardschrift
         numbers=left,               % Ort der Zeilennummern
         numberstyle=\tiny,          % Stil der Zeilennummern
         stepnumber=2,               % Abstand zwischen den Zeilennummern
         numbersep=5pt,              % Abstand der Nummern zum Text
         tabsize=2,                  % Groesse von Tabs
         extendedchars=true,         %
         breaklines=true,            % Zeilen werden Umgebrochen
         keywordstyle=[1]\textbf,    % Stil der Keywords
 %        keywordstyle=[2]\textbf,    %
 %        keywordstyle=[3]\textbf,    %
 %        keywordstyle=[4]\textbf,    %
         stringstyle=\color{stringcolor}, % Farbe der String
         showspaces=false,           % Leerzeichen anzeigen ?
         showtabs=false,             % Tabs anzeigen ?
         showstringspaces=false     % Leerzeichen in Strings anzeigen ?
         %frame=single
 }
 \lstloadlanguages{% Check Dokumentation for further languages ...
         Java
         %Pascal
         %C
         %C++
         %XML
         %HTML
 }
\lstset{
    columns=fullflexible,
    frame=shadowbox,
    rulesepcolor=\color{black},
    framexleftmargin=4mm,
    xleftmargin=5mm,
    xrightmargin=1mm,
    numbers=left,
    stepnumber=1,
    numberblanklines=true,
    captionpos=t
}
\lstdefinestyle{simple}{
    frame=none,
    framexleftmargin=1mm,
    xleftmargin=0mm,
    numbers=none
}
\lstdefinestyle{nonumbers}{
    framexleftmargin=1mm,
    xleftmargin=0mm,
    numbers=none
}

%% Seitenlayout ==========================================================
%
% Layout laden um im Dokument den Befehl \layout nutzen zu koennen
%%% Doc: no documentation
%\usepackage[verbose]{layout}
%

% Layout mit 'geometry'
%%% Doc: ftp://tug.ctan.org/pub/tex-archive/macros/latex/contrib/geometry/manual.pdf
% \usepackage{geometry}

\IfPackageLoaded{geometry}{%
\geometry{%
%%% Paper Groesse
   a4paper, % Andere a0paper, a1paper, a2paper, a3paper, , a5paper, a6paper,
            % b0paper, b1paper, b2paper, b3paper, b4paper, b5paper, b6paper
            % letterpaper, executivepaper, legalpaper
   %screen,  % a special paper size with (W,H) = (225mm,180mm)
   %paperwidth=,
   %paperheight=,
   %papersize=, %{ width , height }
   %landscape,  % Querformat
   portrait,    % Hochformat
%%% Koerper Groesse
   %hscale=,      % ratio of width of total body to \paperwidth
                  % hscale=0.8 is equivalent to width=0.8\paperwidth. (0.7 by default)
   %vscale=,      % ratio of height of total body to \paperheight
                  % vscale=0.9 is equivalent to height=0.9\paperheight.
   %scale=,       % ratio of total body to the paper. scale={ h-scale , v-scale }
   %totalwidth=,    % width of total body % (Generally, width >= textwidth)
   %totalheight=,   % height of total body, excluding header and footer by default
   %total=,        % total={ width , height }
   %textwidth=,    % modifies \textwidth, the width of body
   %textheight=,   % modifies \textheight, the height of body
   %body=,        % { width , height } sets both \textwidth and \textheight of the body of page.
   lines=45,       % enables users to specify \textheight by the number of lines.
   %includehead,  % includes the head of the page, \headheight and \headsep, into total body.
   %includefoot,  % includes the foot of the page, \footskip, into body.
   %includeheadfoot, % sets both includehead and includefoot to true
   %includemp,    % includes the margin notes, \marginparwidth and \marginparsep, into body
   %includeall,   % sets both includeheadfoot and includemp to true.
   %ignorehead,   % disregards the head of the page, headheight and headsep in determining vertical layout
   ignorefoot,   % disregards the foot of page, footskip, in determining vertical layout
   %ignoreheadfoot, % sets both ignorehead and ignorefoot to true.
   %ignoremp,     % disregards the marginal notes in determining the horizontal margins
   ignoreall,     % sets both ignoreheadfoot and ignoremp to true
   heightrounded, % This option rounds \textheight to n-times (n: an integer) of \baselineskip
   %hdivide=,     % { left margin , width , right margin }
                  % Note that you should not specify all of the three parameters
   %vdivide=,     % { top margin , height , bottom margin }
   %divide=,      % ={A,B,C} %  is interpreted as hdivide={A,B,C} and vdivide={A,B,C}.
%%% Margin
   %left=80pt,        % left margin (for oneside) or inner margin (for twoside) of total body
                  % alias: lmargin, inner
   %right=,       % right or outer margin of total body
                  % alias: rmargin outer
   %top=,         % top margin of the page.
                  % Alias : tmargin
   %bottom=,      % bottom margin of the page
                  % Alias : bmargin
   %hmargin=,     % left and right margin. hmargin={ left margin , right margin }
   %vmargin=,     % top and bottom margin. vmargin={ top margin , bottom margin }
   %margin=,      % margin={A,B} is equivalent to hmargin={A,B} and vmargin={A,B}
   %hmarginratio, % horizontal margin ratio of left (inner) to right (outer).
   %vmarginratio, % vertical margin ratio of top to bottom.
   %marginratio,  % marginratio={ horizontal ratio , vertical ratio }
   %hcentering,   % sets auto-centering horizontally and is equivalent to hmarginratio=1:1
   %vcentering,   % sets auto-centering vertically and is equivalent to vmarginratio=1:1
   %centering,    % sets auto-centering and is equivalent to marginratio=1:1
   %twoside,       % switches on twoside mode with left and right margins swapped on verso pages.
   %asymmetric,   % implements a twosided layout in which margins are not swapped on alternate pages
                  % and in which the marginal notes stay always on the same side.
   bindingoffset=5mm,  % removes a specified space for binding
%%% Dimensionen
   %headheight=,  % Alias:  head
   %headsep=,     % separation between header and text
   %footskip=,    % distance separation between baseline of last line of text and baseline of footer
                  % Alias: foot
   %nohead,       % eliminates spaces for the head of the page
                  % equivalent to both \headheight=0pt and \headsep=0pt.
   %nofoot,       % eliminates spaces for the foot of the page
                  % equivalent to \footskip=0pt.
   %noheadfoot,   % equivalent to nohead and nofoot.
   footnotesep=0pt, % changes the dimension \skip\footins,.
                  % separation between the bottom of text body and the top of footnote text
   marginparwidth=80pt, % width of the marginal notes
                  % Alias: marginpar
   %marginparsep=,% separation between body and marginal notes.
   %nomarginpar,  % shrinks spaces for marginal notes to 0pt
   %columnsep=,   % the separation between two columns in twocolumn mode.
   %hoffset=,
   %voffset=,
   %offset=,      % horizontal and vertical offset.
                  % offset={ hoffset , voffset }
   %twocolumn,    % twocolumn=false denotes onecolumn
   textwidth=400pt,   % sets \textwidth directly
   %textheight=,  % sets \textheight directly
   %reversemp,    % makes the marginal notes appear in the left (inner) margin
                  % Alias: reversemarginpar
}
} % Endif

% - Anzeigen des Layouts -
\IfPackageLoaded{geometry}{%
   %\geometry{showframe}
}

% Layout mit 'typearea'
%%% Doc: ftp://tug.ctan.org/pub/tex-archive/macros/latex/contrib/koma-script/scrguide.pdf

\IfPackageLoaded{typearea}{% Wenn typearea geladen ist
   \IfPackageNotLoaded{geometry}{% aber nicht geometry
      \typearea[current]{last}
   }
}

% BCOR
%    current  % Satzspiegelberechnung mit dem aktuell gültigen BCOR-Wert erneut
%             % durchführen.
% DIV
%    calc     % Satzspiegelberechnung einschließlich Ermittlung eines guten
%             % DIV-Wertes erneut durchführen.
%    classic  % Satzspiegelberechnung nach dem
%             % mittelalterlichen Buchseitenkanon
%             % (Kreisberechnung) erneut durchführen.
%    current  % Satzspiegelberechnung mit dem aktuell gültigen DIV-Wert erneut
%             % durchführen.
%    default  % Satzspiegelberechnung mit dem Standardwert für das aktuelle
%             % Seitenformat und die aktuelle Schriftgröße erneut durchführen.
%             % Falls kein Standardwert existiert calc anwenden.
%    last     % Satzspiegelberechnung mit demselben DIV -Argument, das beim
%             % letzten Aufruf angegeben wurde, erneut durchführen


%\usepackage[colorgrid,texcoord,gridunit=mm]{showframe}

\raggedbottom     % Variable Seitenhoehen zulassen

% Farben ================================================================

\IfDefined{definecolor}{%

% Farbe der Ueberschriften
%\definecolor{sectioncolor}{RGB}{0, 51, 153} % Blau
%\definecolor{sectioncolor}{RGB}{0, 25, 152}    % Blau (dunkler))
\definecolor{sectioncolor}{RGB}{0, 0, 0}    % Schwarz
%
% Farbe des Textes
\definecolor{textcolor}{RGB}{0, 0, 0}        % Schwarz
%
% Farbe fuer grau hinterlegte Boxen (fuer Paket framed.sty)
\definecolor{shadecolor}{gray}{0.90}

% Farben fuer die Links im PDF
\definecolor{pdfurlcolor}{rgb}{0,0,0.6}
\definecolor{pdffilecolor}{rgb}{0.7,0,0}
\definecolor{pdflinkcolor}{rgb}{0,0,0.0}
\definecolor{pdfcitecolor}{rgb}{0,0,0.6}

%% PDF-Linkfarben auf schwarz für den Druck:
% \definecolor{pdfurlcolor}{rgb}{0,0,0}
% \definecolor{pdffilecolor}{rgb}{0,0,0}
% \definecolor{pdflinkcolor}{rgb}{0,0,0}
% \definecolor{pdfcitecolor}{rgb}{0,0,0}


% Farben fuer Listings
\colorlet{stringcolor}{green!40!black!100}
\colorlet{commencolor}{blue!0!black!100}

} % Endif

%% Aussehen der URLS======================================================
%%% Doc: ftp://tug.ctan.org/pub/tex-archive/macros/latex/contrib/hyperref/doc/manual.pdf
\hypersetup{
   % Farben fuer die Links
   pdfpagelabels,
   colorlinks=true,         % Links erhalten Farben statt Kaeten
   urlcolor=pdfurlcolor,    % \href{...}{...} external (URL)
   filecolor=pdffilecolor,  % \href{...} local file
   linkcolor=pdflinkcolor,  %\ref{...} and \pageref{...}
   citecolor=pdfcitecolor,  %
   % Links
   raiselinks=true,			 % calculate real height of the link
   breaklinks,              % Links berstehen Zeilenumbruch
   %backref=page,            % Backlinks im Literaturverzeichnis (section, slide, page, none)
   %pagebackref=true,        % Backlinks im Literaturverzeichnis mit Seitenangabe
   verbose,
   hyperindex=true,         % backlinkex index
   linktocpage=true,        % Inhaltsverzeichnis verlinkt Seiten
   hyperfootnotes=false,     % Keine Links auf Fussnoten
   % Bookmarks
   bookmarks=true,          % Erzeugung von Bookmarks fuer PDF-Viewer
   bookmarksopenlevel=1,    % Gliederungstiefe der Bookmarks
   bookmarksopen=true,      % Expandierte Untermenues in Bookmarks
   bookmarksnumbered=true,  % Nummerierung der Bookmarks
   bookmarkstype=toc,       % Art der Verzeichnisses
   % Anchors
   plainpages=false,        % Anchors even on plain pages ?
   pageanchor=true,         % Pages are linkable
   % PDF Informationen
   pdftitle={},             % Titel
   pdfauthor={},            % Autor
   pdfcreator={LaTeX, hyperref, KOMA-Script}, % Ersteller
   %pdfproducer={pdfeTeX 1.10b-2.1} %Produzent
   pdfdisplaydoctitle=true, % Dokumententitel statt Dateiname im Fenstertitel
   pdfstartview=FitH,       % Dokument wird Fit Width geaefnet
   pdfpagemode=UseOutlines, % Bookmarks im Viewer anzeigen
   pdfpagelabels=true,           % set PDF page labels
   pdfpagelayout=TwoPageRight, % zweiseitige Darstellung: ungerade Seiten
   									 % rechts im PDF-Viewer
   %pdfpagelayout=SinglePage, % einseitige Darstellung
}

%fuer URL (nur wenn url geladen ist)
\IfDefined{urlstyle}{
	\urlstyle{tt} %sf
}

%% Kopf und Fusszeilen====================================================
%%% Doc: ftp://tug.ctan.org/pub/tex-archive/macros/latex/contrib/koma-script/scrguide.pdf

\usepackage[%
   % headtopline,
   % plainheadtopline,
   % headsepline,
   % plainheadsepline,
   % footsepline,
   % plainfootsepline,
   % footbotline,
   % plainfootbotline,
   % ilines,
   % clines,
   % olines,
   automark,
   % autooneside,% ignore optional argument in automark at oneside
   komastyle,
   % standardstyle,
   % markuppercase,
   % markusedcase,
   nouppercase,
]{scrpage2}

%\usepackage[%
%   automark,         % automatische Aktualisierung der Kolumnentitel
%   nouppercase,      % Grossbuchstaben verhindern
%   %markuppercase    % Grossbuchstaben erzwingen
%   %markusedcase     % vordefinierten Stil beibehalten
%   %komastyle,       % Stil von Koma Script
%   %standardstyle,   % Stil der Standardklassen
%]{scrpage2}

\IfElseChapterDefined{%
   \pagestyle{scrheadings} % Seite mit Headern
}{
   \pagestyle{scrplain} % Seiten ohne Header
}
%\pagestyle{empty} % Seiten ohne Header
%
% loescht voreingestellte Stile
\clearscrheadings
\clearscrplain
%
% Was steht wo...
\IfElseChapterDefined{
   % Oben aussen: Kapitel und Section
   % Unten aussen: Seitenzahl
   % \ohead{\headmark} % Oben außen: Setzt Kapitel und Section automatisch
   % \ofoot[\pagemark]{\pagemark}
   % oder...
   % Oben aussen: Seitenzahlen
   % Oben innen: Kapitel und Section
   \ohead{\pagemark}
   \ihead{\headmark}
   \ofoot[\pagemark]{} % Außen unten: Seitenzahlen bei plain
}{
   \cfoot[\pagemark]{\pagemark} % Mitte unten: Seitenzahlen bei plain
}
% Vollstaendige Liste der moeglichen Positionierungen
% \lehead[scrplain-links-gerade]{scrheadings-links-gerade}
% \cehead[scrplain-mittig-gerade]{scrheadings-mittig-gerade}
% \rehead[scrplain-rechts-gerade]{scrheadings-rechts-gerade}
% \lefoot[scrplain-links-gerade]{scrheadings-links-gerade}
% \cefoot[scrplain-mittig-gerade]{scrheadings-mittig-gerade}
% \refoot[scrplain-rechts-gerade]{scrheadings-rechts-gerade}
% \lohead[scrplain-links-ungerade]{scrheadings-links-ungerade}
% \cohead[scrplain-mittig-ungerade]{scrheadings-mittig-ungerade}
% \rohead[scrplain-rechts-ungerade]{scrheadings-rechts-ungerade}
% \lofoot[scrplain-links-ungerade]{scrheadings-links-ungerade}
% \cofoot[scrplain-mittig-ungerade]{scrheadings-mittig-ungerade}
% \rofoot[scrplain-rechts-ungerade]{scrheadings-rechts-ungerade}
% \ihead[scrplain-innen]{scrheadings-innen}
% \chead[scrplain-zentriert]{scrheadings-zentriert}
% \ohead[scrplain-außen]{scrheadings-außen}
% \ifoot[scrplain-innen]{scrheadings-innen}
% \cfoot[scrplain-zentriert]{scrheadings-zentriert}
% \ofoot[scrplain-außen]{scrheadings-außen}


%\usepackage{lastpage} % Stellt 'LastPage' zur Verfuegung
%\cfoot[Seite \pagemark~von \pageref{LastPage}]{} % Seitenzahl von Anzahl Seiten

% Angezeigte Abschnitte im Header
\IfElseChapterDefined{
   \automark[section]{chapter} %[rechts]{links}
}{
   \automark[subsection]{section} %[rechts]{links}
}
%
% Linien (moegliche Kombination mit Breiten)
\IfChapterDefined{
   %\setheadtopline{}     % modifiziert die Parameter fuer die Linie ueber dem
   							  %	 Seitenkopf
   %\setheadsepline{.4pt}[\color{black}]
                         % modifiziert die Parameter fuer die Linie zwischen
                         % Kopf und Textkörper
   %\setfootsepline{}    % modifiziert die Parameter fuer die Linie zwischen
   							 % Text und Fuß
   %\setfootbotline{}    % modifiziert die Parameter fuer die Linie unter dem
   							 % Seitenfuss
}



% Groesse des Headers

% Breite von Kopf und Fusszeile einstellen
% \setheadwidth[Verschiebung]{Breite}
% \setfootwidth[Verschiebung]{Breite}
% mögliche Werte
% paper - die Breite des Papiers
% page - die Breite der Seite
% text - die Breite des Textbereichs
% textwithmarginpar - die Breite des Textbereichs inklusive dem Seitenrand
% head - die aktuelle Breite des Seitenkopfes
% foot - die aktuelle Breite des Seitenfusses
\setheadwidth[0pt]{text}
\setfootwidth[0pt]{text}

%\setheadwidth[]%
%{%
%   paper % width of paper
%   page  % width of page (paper - BCOR)
%   text  % \textwidth
%   textwithmarginpar % width of text plus margin
%   head  % current width of head
%   foot  % current width of foot
%}%


%% Fussnoten =============================================================
% Keine hochgestellten Ziffern in der Fussnote (KOMA-Script-spezifisch):
\deffootnote{1.5em}{1em}{\makebox[1.5em][l]{\thefootnotemark}}
\addtolength{\skip\footins}{\baselineskip} % Abstand Text <-> Fussnote

\setlength{\dimen\footins}{10\baselineskip} % Beschraenkt den Platz von Fussnoten auf 10 Zeilen

\interfootnotelinepenalty=10000 % Verhindert das Fortsetzen von
                                % Fussnoten auf der gegenüberligenden Seite


%% Schriften (Sections )==================================================

\IfElsePackageLoaded{fourier}{
   \newcommand\SectionFontStyle{\rmfamily}
}{
   \newcommand\SectionFontStyle{\sffamily}
}

% -- Koma Schriften --
\IfChapterDefined{%
   \setkomafont{chapter}{\huge\SectionFontStyle}    % Chapter
   %\setkomafont{chapter}{\bfseries\Huge\SectionFontStyle}    % Chapter
   %\addtokomafont{chapterentry}{\bfseries}
}

\addtokomafont{paragraph}{\bfseries}
\setkomafont{sectioning}{\SectionFontStyle}
%\setkomafont{part}{\usekomafont{sectioning}}
%\setkomafont{section}{\usekomafont{sectioning}}
%\setkomafont{subsection}{\usekomafont{sectioning}}
%\setkomafont{subsubsection}{\usekomafont{sectioning}}
%\setkomafont{paragraph}{\usekomafont{sectioning}}
%\setkomafont{subparagraph}{\usekomafont{sectioning}}

\setkomafont{descriptionlabel}{\itshape}

%\setkomafont{caption}{\normalfont}
%\setkomafont{captionlabel}{\normalfont}

%\setkomafont{dictum}{}
%\setkomafont{dictumauthor}{}
%\setkomafont{dictumtext}{}
%\setkomafont{disposition}{}
%\setkomafont{footnote}{}
%\setkomafont{footnotelabel}{}
%\setkomafont{footnotereference}{}
%\setkomafont{minisec}{}

%\setkomafont{partnumber}{\bfseries\SectionFontStyle}
%\setkomafont{partentrynumber}{}
%\setkomafont{chapterentrypagenumber}{}
%\setkomafont{sectionentrypagenumber}{}

\setkomafont{pageheadfoot}{\normalfont\normalcolor\small\sffamily}
\setkomafont{pagenumber}{\bfseries\usekomafont{sectioning}}

%\setkomafont{partentry}{\usekomafont{sectioning}\large}
%\setkomafont{chapterentry}{\usekomafont{sectioning}}
%\setkomafont{sectionentry}{\usekomafont{sectioning}}

%%% --- Titlepage ---
%\setkomafont{subject}{}
%\setkomafont{subtitle}{}
%\setkomafont{title}{}



\addtokomafont{sectioning}{\color{sectioncolor}} % Farbe der Ueberschriften
\IfChapterDefined{%
	\addtokomafont{chapter}{\color{sectioncolor}} % Farbe der Ueberschriften
}
\renewcommand*{\raggedsection}{\raggedright} % Titelzeile linksbuendig, haengend
%
%% UeberSchriften (Chapter und Sections) =================================

%%% Remove Space above Chapter.
%%% (NOT recommanded!)
%% Space above Chapter Title
% \renewcommand*{\chapterheadstartvskip}{\vspace{1\baselineskip}}%
%% Space below Chapter Title
% \renewcommand*{\chapterheadendvskip}{\vspace{0.5\baselineskip}}%

% -- Ueberschriften komlett Umdefinieren --
%%% Doc: ftp://tug.ctan.org/pub/tex-archive/macros/latex/contrib/titlesec/titlesec.pdf
%\usepackage{titlesec}

% -- Section Aussehen veraendern --
% --------------------------------
%% -> Section mit Unterstrich
% \titleformat{\section}
%   [hang]%[frame]display
%   {\usekomafont{sectioning}\Large}
%  {\thesection}
%   {6pt}
%   {}
%   [\titlerule \vspace{0.5\baselineskip}]
% --------------------------------

% -- Chapter Aussehen veraendern --
% --------------------------------
%--> Box mit (Kapitel + Nummer ) +  Name
% \titleformat{\chapter}[display]     % {command}[shape]
%   {\usekomafont{chapter}\filcenter} % format
%   {                                 % label
%   {\fcolorbox{black}{shadecolor}{
%   {\huge\chaptertitlename\mbox{\hspace{1mm}}\thechapter}
%   }}}
%   {1pc}                             % sep (from chapternumber)
%   {\vspace{1pc}}                    % {before}[after] (before chaptertitle and after)
% --------------------------------
%--> Kapitel + Nummer + Trennlinie + Name + Trennlinie
%\titleformat{\chapter}[display]	% {command}[shape]
%  {\usekomafont{chapter}\Large \color{black}}	% format
%  {   										% label
%  \LARGE\MakeUppercase{\chaptertitlename} \Huge \thechapter \filright%
%  }%}
%  {1pt}										% sep (from chapternumber)
%  {\titlerule \vspace{0.9pc} \filright \color{sectioncolor}}   % {before}[after] (before chaptertitle and after)
%  [\color{black} \vspace{0.9pc} \filright {\titlerule}]


%%% Doc: No documentation
% Indent first paragraph after section header
% \usepackage{indentfirst}

%% Captions (Schrift, Aussehen) ==========================================

% % Folgende Befehle werden durch das Paket caption und subfig ersetzt !
% \setcapindent{1em} % Einrueckung der Beschriftung
% \setkomafont{caption}{\color{black}\small\sffamily\RaggedRight}  % Schrift fuer Caption
% \setkomafont{captionlabel}{\color{black}\small}   % Schrift fuer 'Abbildung' usw.

%%% Doc: ftp://tug.ctan.org/pub/tex-archive/macros/latex/contrib/caption/caption.pdf
\usepackage{caption}
% Aussehen der Captions
\captionsetup{
   margin = 10pt,
   font = {small,rm},
   labelfont = {small,bf},
   format = plain, % oder 'hang'
   indention = 0em,  % Einruecken der Beschriftung
   labelsep = colon, %period, space, quad, newline
   justification = RaggedRight, % justified, centering
   singlelinecheck = true, % false (true=bei einer Zeile immer zentrieren)
   position = bottom %top
}
%%% Bugfix Workaround
\DeclareCaptionOption{parskip}[]{}
\DeclareCaptionOption{parindent}[]{}

% Aussehen der Captions fuer subfigures (subfig-Paket)
\IfPackageLoaded{subfig}{
 \captionsetup[subfloat]{%
   margin = 10pt,
   font = {small,rm},
   labelfont = {small,bf},
   format = plain, % oder 'hang'
   indention = 0em,  % Einruecken der Beschriftung
   labelsep = space, %period, space, quad, newline
   justification = RaggedRight, % justified, centering
   singlelinecheck = true, % false (true=bei einer Zeile immer zentrieren)
   position = bottom, %top
   labelformat = parens % simple, empty % Wie die Bezeichnung gesetzt wird
 }
}

% Aendern der Bezeichnung fuer Abbildung und Tabelle
% \addto\captionsngerman{% "captionsgerman" fuer alte  Rechschreibung
%   \renewcommand{\figurename}{Abb.}%
%   \renewcommand{\tablename}{Tab.}%
% }

%% Inhaltsverzeichnis (Schrift, Aussehen) sowie weitere Verzeichnisse ====

\setcounter{secnumdepth}{2}    % Abbildungsnummerierung mit groesserer Tiefe
\setcounter{tocdepth}{2}		 % Inhaltsverzeichnis mit groesserer Tiefe
%

% Inhalte von List of Figures
\IfPackageLoaded{subfig}{
	\setcounter{lofdepth}{1}  %1 = nur figures, 2 = figures + subfigures
}

% -------------------------------------------------------

% Aussehen des Inhaltsverzeichnisses: tocloft
%%% Doc: ftp://tug.ctan.org/pub/tex-archive/macros/latex/contrib/tocloft/tocloft.pdf
%% Laden mit Option subfigure in Abhaengigkeit vom Paket subfigure und subfig
% \IfElsePackageLoaded{subfig}
% 	% IF subfig
% 	{\usepackage[subfigure]{tocloft}}{
% 	% ELSE
% 	\IfElsePackageLoaded{subfigure}
% 		% IF subfigure
% 		{\usepackage[subfigure]{tocloft}}
% 	   % Else (No subfig nor subfigure)
% 		{\usepackage{tocloft}}
% 	}
%
% %TOCLOFT zerstoert Layout der Ueberschriften von TOC, LOT, LOF
% \IfPackageLoaded{tocloft}{
% %
% %%%% Layout Matthias Pospiech (alles serifenlos)
% \IfChapterDefined{%
% 	\renewcommand{\cftchappagefont}{\bfseries\sffamily}  % Kapitel Seiten Schrift
% 	\renewcommand{\cftchapfont}{\bfseries\sffamily}      % Kapitel Schrift
% }
% \renewcommand{\cftsecpagefont}{\sffamily}            % Section Seiten Schrift
% \renewcommand{\cftsubsecpagefont}{\sffamily}         % Subsectin Seiten Schrift
% \renewcommand{\cftsecfont}{\sffamily}                % Section Schrift
% \renewcommand{\cftsubsecfont}{\sffamily}             % Subsection Schrift
%
% %%%% Layout aus Typokurz:
% % % Seitenzahlen direkt hinter TOC-Eintrag:
% % % Ebene \chapter
% % \renewcommand{\cftchapleader}{}
% % \renewcommand{\cftchapafterpnum}{\cftparfillskip}
% % % Ebene \section
% % \renewcommand{\cftsecleader}{}
% % \renewcommand{\cftsecafterpnum}{\cftparfillskip}
% % % Ebene \subsection
% % \renewcommand{\cftsubsecleader}{}
% % \renewcommand{\cftsubsecafterpnum}{\cftparfillskip}
% % % Abstaende vor Eintraegen im TOC verkleinern
% % \setlength{\cftbeforesecskip}{.4\baselineskip}
% % \setlength{\cftbeforesubsecskip}{.1\baselineskip}
% }
% % Ende tocloft Einstellungen --------------

%%% Doc: ftp://tug.ctan.org/pub/tex-archive/macros/latex/contrib/ms/multitoc.dvi
% TOC in mehreren Spalten setzen
%\usepackage[toc]{multitoc}

% -------------------------

%%Schriften fuer Minitoc (Inhaltsverzeichnis vor jedem Kapitel)
%%% Doc: ftp://tug.ctan.org/pub/tex-archive/macros/latex/contrib/minitoc/minitoc.pdf
%\IfChapterDefined{%
% \usepackage{minitoc}
% \setlength{\mtcindent}{0em} % default: 24pt
% \setcounter{minitocdepth}{2}
% \setlength{\mtcskipamount}{\bigskipamount}
% \mtcsettitlefont{minitoc}{\normalsize\SectionFontStyle}
% \mtcsetfont{minitoc}{*}{\small\SectionFontStyle} %\color{textcolor}
% \mtcsetfont{minitoc}{section}{\small\SectionFontStyle}
% \mtcsetfont{minitoc}{subsection}{\small\SectionFontStyle}
% \mtcsetfont{minitoc}{subsubsection}{\small\SectionFontStyle}
%}
%{%
%\usepackage{minitoc}
%\setlength{\stcindent}{0pt} %default
%\setcounter{secttocdepth}{2} %default
%\mtcsettitlefont{secttoc}{\SectionFontStyle}
%\mtcsetfont{secttoc}{*}{\small\SectionFontStyle}%
%\mtcsetfont{secttoc}{subsection}{\small\SectionFontStyle}
%\mtcsetfont{secttoc}{subsubsection}{\small\SectionFontStyle}
%}

% Packages that MUST be loaded before minitoc !
% hyperref, caption, sectsty, varsects, fncychap, hangcaption, quotchap, romannum, sfheaders, alnumsec, captcont


%% Index & Co. ===========================================================
% gibts dafuer noch eine sauberere Loesung ?
%%%%%%%% Index zweispaltig %%%%%%%
% \makeatletter
% \renewenvironment{theindex}{%
% \setlength{\columnsep}{2em}
% \begin{multicols}{2}[\section*{\indexname}]
% \parindent\z@
% \parskip\z@ \@plus .3\p@\relax
% \let\item\@idxitem}%
% {\end{multicols}\clearpage}
% \makeatother
%%%%%%%%%%%%%%%%%%%%%%%%%%%%%%%%%%
