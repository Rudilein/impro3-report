% ------------------------------------------------------------------------
% LaTeX - Preambel  ******************************************************
% ------------------------------------------------------------------------
% von: Matthias Pospiech
% ========================================================================

% Strukturierung dieser Praeambel:
%    1.  Pakete die vor anderen geladen werden m�ssen
%        (calc, babel, xcolor, graphicx, amsmath, pst-pdf, ragged2e, ...)
%    2.  Schriften
%    3.  Mathematik (mathtools, fixmath, onlyamsmath, braket,
%        cancel, empheq, exscale, icomma, ...)
%    4.  Tabellen (booktabs, multirow, dcolumn, tabularx, ltxtable, supertabular)
%    5.  Text
%        5.1 Auszeichnungen (ulem, soul, url)
%        5.2 Fussnoten (footmisc)
%        5.3 Verweise (varioref)
%        5.4 Listen (enumitem, paralist, declist)
%    6.  Zitieren (csquotes, jurabib, natbib)
%    7.  PDF (microtype, hyperref, backref, hypcap, pdfpages
%    8.  Graphiken (float, flafter, placeins, subfig, wrapfig,
%        floatflt, picins, psfrag, sidecap, pict2e, curve2e)
%    9.  Sonstiges (makeidx, isodate, numprint, nomencl, acronym)
%    10. Verbatim (upquote, verbatim, fancyvrb, listings, examplep)
%    11. Wissenschaft (units)
%    12. Fancy Stuff
%    13. Layout
%       13.1.  Diverse Pakete und Einstellungen (multicol, ellipsis)
%       13.2.  Zeilenabstand (setspace)
%       13.3.  Seitenlayout (typearea, geometry)
%       13.4.  Farben
%       13.5.  Aussehen der URLS
%       13.6.  Kopf und Fusszeilen (scrpage2)
%       13.7.  Fussnoten
%       13.8.  Schriften (Sections )
%       13.9.  UeberSchriften (Chapter und Sections) (titlesec, indentfirst)
%       13.10. Captions (Schrift, Aussehen)
%              (caption, subfig, capt-of, mcaption, tocloft, multitoc, minitoc)
%    14.  Auszufuehrende Befehle


% ~~~~~~~~~~~~~~~~~~~~~~~~~~~~~~~~~~~~~~~~~~~~~~~~~~~~~~~~~~~~~~~~~~~~~~~~
% Einige Pakete muessen unbedingt vor allen anderen geladen werden
% ~~~~~~~~~~~~~~~~~~~~~~~~~~~~~~~~~~~~~~~~~~~~~~~~~~~~~~~~~~~~~~~~~~~~~~~~

%
%%% Doc: www.cs.brown.edu/system/software/latex/doc/calc.pdf
% Calculation with LaTeX
\usepackage{calc}

%%% Doc: ftp://tug.ctan.org/pub/tex-archive/macros/latex/required/babel/babel.pdf
% Languagesetting
\usepackage[
%	german,
	ngerman,
	english,
%	french,
]{babel}

%%% Doc: ftp://tug.ctan.org/pub/tex-archive/macros/latex/contrib/xcolor/xcolor.pdf
% Farben
% Incompatible: Do not load when using pstricks !
\usepackage[
	table % Load for using rowcolors command in tables
]{xcolor}


%%% Doc: ftp://tug.ctan.org/pub/tex-archive/macros/latex/required/graphics/grfguide.pdf
% Bilder
\usepackage[%
	%final,
	%draft % do not include images (faster)
]{graphicx}

%%% Doc: ftp://tug.ctan.org/pub/tex-archive/macros/latex/contrib/oberdiek/epstopdf.pdf
%% If an eps image is detected, epstopdf is automatically called to convert it to pdf format.
%% Requires: graphicx loaded
\usepackage{epstopdf}


%%% Doc: ftp://tug.ctan.org/pub/tex-archive/macros/latex/required/amslatex/math/amsldoc.pdf
% Amsmath - Mathematik Basispaket
%
% fuer pst-pdf displaymath Modus vor pst-pdf benoetigt.
\usepackage[
   centertags, % (default) center tags vertically
   %tbtags,    % 'Top-or-bottom tags': For a split equation, place equation numbers level
               % with the last (resp. first) line, if numbers are on the right (resp. left).
   sumlimits,  %(default) Place the subscripts and superscripts of summation
               % symbols above and below
   %nosumlimits, % Always place the subscripts and superscripts of summation-type
               % symbols to the side, even in displayed equations.
   intlimits,  % Like sumlimits, but for integral symbols.
   %nointlimits, % (default) Opposite of intlimits.
   namelimits, % (default) Like sumlimits, but for certain 'operator names' such as
               % det, inf, lim, max, min, that traditionally have subscripts placed underneath
               % when they occur in a displayed equation.
   %nonamelimits, % Opposite of namelimits.
   %leqno,     % Place equation numbers on the left.
   %reqno,     % Place equation numbers on the right.
   %fleqn,     % Position equations at a fixed indent from the left margin
   			  % rather than centered in the text column.
]{amsmath} %
% eqnarray nicht zusammen mit amsmath benutzen, siehe l2tabu.pdf f�r
% Hintergruende.

\usepackage{amsthm}

%%% Doc: http://www.ctan.org/tex-archive/macros/latex/contrib/pst-pdf/pst-pdf-DE.pdf
% Used to automatically integrate eps graphics in an pdf document using pdflatex.
% Requires ps4pdf macro !!!
% Download macro from http://www.ctan.org/tex-archive/macros/latex/contrib/pst-pdf/scripts/
%
%\usepackage[%
%   %active,       % Aktiviert den Extraktionsmodus (DVI-Ausgabe). Die explizite Angabe ist
%                  % normalerweise unn�tig (Standard im LATEX-Modus).
%   %inactive,     % Das Paket wird deaktiviert, Zu�tzlich werden die Pakete pstricks und
%                  % graphicx geladen
%   nopstricks,    % Das Paket pstricks wird nicht geladen.
%   %draft,        % Im pdfLATEX-Modus werden aus der Containerdatei eingef�gte Grafiken nur
%                  % als Rahmen dargestellt.
%   %final,        % Im pdfLATEX-Modus werden aus der Containerdatei eingef�gte Grafiken
%                  % vollst�ndig dargestellt (Standard).
%   %tightpage,    % Die Abmessung Grafiken in der Containerdatei entsprechen denen der
%                  % zugeh�rigen TEX-Boxen (Standard).
%   %notightpage,  % die Grafiken in der Containerdatei nehmen
%                  % mindestens die Gr��e des gesamten Blattes einnehmen.
%   %displaymath,  % Es werden zus�tzlich die mathematischen Umgebungen displaymath,
%                  % eqnarray und $$ extrahiert und im pdf-Modus als Grafik eingef�gt.
%]{pst-pdf}
%
% Notwendiger Bugfix f�r natbib Paket bei Benutzung von pst-pdf (Version <= v1.1o)
\IfPackageLoaded{pst-pdf}{
   \providecommand\makeindex{}
   \providecommand\makeglossary{}
}{}


%% Doc: ftp://tug.ctan.org/pub/tex-archive/graphics/pstricks/README
% load before graphicx
% \usepackage{pstricks}
% \usepackage{pst-plot, pst-node, pst-coil, pst-eps}

% This package implements a workaround for the LaTeX bug that marginpars
% sometimes appear on the wrong margin.
% \usepackage{mparhack}
% in some case this causes an error in the index together with package pdfpages
% the reason is unkown. Therefore I recommend to use the margins of marginnote

%% Doc: ftp://tug.ctan.org/pub/tex-archive/macros/latex/contrib/marginnote/marginnote.pdf
% Summary description: marginnote allows margin note, where \marginpar fails
\usepackage{marginnote}


%% Doc: (inside relsize.sty )
%% ftp://tug.ctan.org/pub/tex-archive/macros/latex/contrib/misc/relsize.sty
%  Set the font size relative to the current font size
\usepackage{relsize}

%% Doc: ftp://tug.ctan.org/pub/tex-archive/macros/latex/contrib/ms/ragged2e.pdf
% Besserer Flatternsatz (Linksbuendig, statt Blocksatz)
\usepackage{ragged2e}

% ~~~~~~~~~~~~~~~~~~~~~~~~~~~~~~~~~~~~~~~~~~~~~~~~~~~~~~~~~~~~~~~~~~~~~~~~
% Fonts Fonts Fonts
% ~~~~~~~~~~~~~~~~~~~~~~~~~~~~~~~~~~~~~~~~~~~~~~~~~~~~~~~~~~~~~~~~~~~~~~~~

\usepackage[T1]{fontenc} % T1 Schrift Encoding
\usepackage{textcomp}	 % Zusatzliche Symbole (Text Companion font extension)

% ~~~~~~~~~~~~~~~~~~~~~~~~~~~~~~~~~~~~~~~~~~~~~~~~~~~~~~~~~~~~~~~~~~~~~~~~
% Math Packages
% ~~~~~~~~~~~~~~~~~~~~~~~~~~~~~~~~~~~~~~~~~~~~~~~~~~~~~~~~~~~~~~~~~~~~~~~~

% *** Mathematik **************************************
%
% amsmath schon vorher geladen da es vor pst-pdf geladen werden muss


%%% Doc: ftp://tug.ctan.org/pub/tex-archive/macros/latex/contrib/mh/doc/mathtools.pdf
% Erweitert amsmath und behebt einige Bugs
\usepackage[fixamsmath,disallowspaces]{mathtools}

%%% Doc: http://www.ctan.org/info?id=fixmath
% LaTeX's default style of typesetting mathematics does not comply
% with the International Standards ISO31-0:1992 to ISO31-13:1992
% which indicate that uppercase Greek letters always be typset
% upright, as opposed to italic (even though they usually
% represent variables) and allow for typsetting of variables in a
% boldface italic style (even though the required fonts are
% available). This package ensures that uppercase Greek be typeset
% in italic style, that upright $\Delta$ and $\Omega$ symbols are
% available through the commands \upDelta and \upOmega; and
% provides a new math alphabet \mathbold for boldface
% italic letters, including Greek.
%\usepackage{fixmath}

%%% Doc: ftp://tug.ctan.org/pub/tex-archive/macros/latex/contrib/onlyamsmath/onlyamsmath.dvi
% Warnt bei Benutzung von Befehlen die mit amsmath inkompatibel sind.
\usepackage[
	all,
	warning
]{onlyamsmath}


%------------------------------------------------------

% -- Vektor fett darstellen -----------------
% \let\oldvec\vec
% \def\vec#1{{\boldsymbol{#1}}} %Fetter Vektor
% \newcommand{\ve}{\vec} %
% -------------------------------------------


%%% Doc: ftp://tug.ctan.org/pub/tex-archive/macros/latex/contrib/misc/braket.sty
\usepackage{braket}  % Quantenmechanik Bracket Schreibweise

%%% Doc: ftp://tug.ctan.org/pub/tex-archive/macros/latex/contrib/misc/cancel.sty
\usepackage{cancel}  % Durchstreichen

%%% Doc: ftp://tug.ctan.org/pub/tex-archive/macros/latex/contrib/mh/doc/empheq.pdf
\usepackage{empheq}  % Hervorheben

%%% Doc: ftp://tug.ctan.org/pub/tex-archive/info/math/voss/mathmode/Mathmode.pdf
%\usepackage{exscale} % Skaliert Mathe-Modus Ausgaben in allen Umgebungen richtig.

%%% Doc: ftp://tug.ctan.org/pub/tex-archive/macros/latex/contrib/was/icomma.dtx
% Erlaubt die Benutzung von Kommas im Mathematikmodus
%\usepackage{icomma}


%%% Doc: http://www.ctex.org/documents/packages/special/units.pdf
% \usepackage[nice]{nicefrac}

%%% Tauschen von Epsilon und andere:
% \let\ORGvarrho=\varrho
% \let\varrho=\rho
% \let\rho=\ORGvarrho
%
\let\ORGvarepsilon=\varepsilon
\let\varepsilon=\epsilon
\let\epsilon=\ORGvarepsilon
%
% \let\ORGvartheta=\vartheta
% \let\vartheta=\theta
% \let\theta=\ORGvartheta
%
% \let\ORGvarphi=\varphi
% \let\varphi=\phi
% \let\phi=\ORGvarphi

% ~~~~~~~~~~~~~~~~~~~~~~~~~~~~~~~~~~~~~~~~~~~~~~~~~~~~~~~~~~~~~~~~~~~~~~~~
% Symbole
% ~~~~~~~~~~~~~~~~~~~~~~~~~~~~~~~~~~~~~~~~~~~~~~~~~~~~~~~~~~~~~~~~~~~~~~~~
%
%%% General Doc: http://www.ctan.org/tex-archive/info/symbols/comprehensive/symbols-a4.pdf
%
%% Symbole f�r Mathematiksatz
%\usepackage{mathrsfs} %% Schreibschriftbuchstaben f�r den Mathematiksatz (nur Gro�buchstaben)
%\usepackage{dsfont}   %% Double Stroke Fonts
%\usepackage[mathcal]{euscript} %% adds euler mathcal font
%\usepackage{amssymb}
%\usepackage[Symbolsmallscale]{upgreek} % upright symbols from euler package [Euler] or Adobe Symbols [Symbol]
%\usepackage[upmu]{gensymb}             % Option upmu

%% Allgemeine Symbole
%\usepackage{wasysym}  %% Doc: http://www.ctan.org/tex-archive/macros/latex/contrib/wasysym/wasysym.pdf
%\usepackage{marvosym} %% Symbole aus der marvosym Schrift
%\usepackage{pifont}   %% ZapfDingbats


% ~~~~~~~~~~~~~~~~~~~~~~~~~~~~~~~~~~~~~~~~~~~~~~~~~~~~~~~~~~~~~~~~~~~~~~~~
% Tables (Tabular)
% ~~~~~~~~~~~~~~~~~~~~~~~~~~~~~~~~~~~~~~~~~~~~~~~~~~~~~~~~~~~~~~~~~~~~~~~~

% Basispaket fuer alle Tabellenfunktionen
% -> wird automatisch durch andere Pakete geladen
% \usepackage{array}
%
% bessere Abstaende innerhalb der Tabelle (Layout))
% -------------------------------------------------
%%% Doc: ftp://tug.ctan.org/pub/tex-archive/macros/latex/contrib/booktabs/booktabs.pdf
\usepackage{booktabs}
%
% Farbige Tabellen
% ----------------
% Das Paket colortbl wird inzwischen automatisch durch xcolor geladen
%
% Erweiterte Funktionen innerhalb von Tabellen
% --------------------------------------------
%%% Doc: ftp://tug.ctan.org/pub/tex-archive/macros/latex/contrib/multirow/multirow.sty
\usepackage{multirow} % Mehrfachspalten
%
%%% Doc: Documentation inside dtx Package
\usepackage{dcolumn}  % Ausrichtung an Komma oder Punkt

%%% Neue Tabellen-Umgebungen:
% ---------------------------
% Spalten automatischer Breite:
%%% Doc: Documentation inside dtx Package
% \usepackage{tabularx}
% -> nach hyperref Laden
% -> wird von ltxtable geladen
% \LoadPackageLater{tabularx}


% Tabellen ueber mehere Seiten
% ----------------------------
%%% Doc: ftp://tug.ctan.org/pub/tex-archive/macros/latex/contrib/carlisle/ltxtable.pdf
% \usepackage{ltxtable} % Longtable + tabularx
                        % (multi-page tables) + (auto-sized columns in a fixed width table)
% -> nach hyperref laden
\LoadPackageLater{ltxtable}


%%% Doc: ftp://tug.ctan.org/pub/tex-archive/macros/latex/contrib/supertabular/supertabular.pdf
%\usepackage{supertabular}


% ~~~~~~~~~~~~~~~~~~~~~~~~~~~~~~~~~~~~~~~~~~~~~~~~~~~~~~~~~~~~~~~~~~~~~~~~
% text related packages
% ~~~~~~~~~~~~~~~~~~~~~~~~~~~~~~~~~~~~~~~~~~~~~~~~~~~~~~~~~~~~~~~~~~~~~~~~

%%% Textverzierungen/Auszeichnungen ======================================
%
%%% Doc: ftp://tug.ctan.org/pub/tex-archive/macros/latex/contrib/misc/ulem.sty
\usepackage[normalem]{ulem}      % Zum Unterstreichen
%%% Doc: ftp://tug.ctan.org/pub/tex-archive/macros/latex/contrib/soul/soul.pdf
\usepackage{soul}		            % Unterstreichen, Sperren
%%% Doc: ftp://tug.ctan.org/pub/tex-archive/macros/latex/contrib/misc/url.sty
\usepackage{url} % Setzen von URLs. In Verbindung mit hyperref sind diese auch aktive Links.

%%% Fussnoten/Endnoten ===================================================
%
%%% Doc: ftp://tug.ctan.org/pub/tex-archive/macros/latex/contrib/footmisc/footmisc.pdf
%
\usepackage[
   bottom,      % Footnotes appear always on bottom. This is necessary
                % especially when floats are used
   stable,      % Make footnotes stable in section titles
   perpage,     % Reset on each page
   %para,       % Place footnotes side by side of in one paragraph.
   %side,       % Place footnotes in the margin
   ragged,      % Use RaggedRight
   %norule,     % suppress rule above footnotes
   multiple,    % rearrange multiple footnotes intelligent in the text.
   %symbol,     % use symbols instead of numbers
]{footmisc}

\renewcommand*{\multfootsep}{,\nobreakspace}

\deffootnote%
   [1em]% width of marker
   {1.5em}% indentation (general)
   {1em}% indentation (par)
   {\textsubscript{\thefootnotemark}}%


%% Einruecken der Fussnote einstellen
%\setlength\footnotemargin{10pt}

%--- footnote counter documentweit durchlaufend ------------------------------
%\usepackage{chngcntr}
%\counterwithout{footnote}{chapter}
%-----------------------------------------------------------------------------

%%% Doc: ftp://tug.ctan.org/pub/tex-archive/macros/latex/contrib/misc/endnotes.sty
%\usepackage{endnotes}
% From the Documentation:
% To turn all the footnotes in your documents into endnotes, say
%
%     \let\footnote=\endnote
%
%  in your preamble, and then add something like
%
%     \newpage
%     \begingroup
%     \parindent 0pt
%     \parskip 2ex
%     \def\enotesize{\normalsize}
%     \theendnotes
%     \endgroup
%
% as the last thing in your document.  (But \theendnotes all
% by itself will work.)

% ~~~~~~~~~~~~~~~~~~~~~~~~~~~~~~~~~~~~~~~~~~~~~~~~~~~~~~~~~~~~~~~~~~~~~~~~
% Pakete zum Zitieren
% ~~~~~~~~~~~~~~~~~~~~~~~~~~~~~~~~~~~~~~~~~~~~~~~~~~~~~~~~~~~~~~~~~~~~~~~~

% Quotes =================================================================
%% Doc: ftp://tug.ctan.org/pub/tex-archive/macros/latex/contrib/csquotes/csquotes.pdf
% Advanced features for clever quotations
\usepackage[%
   babel,            % the style of all quotation marks will be adapted
                     % to the document language as chosen by 'babel'
   german=quotes,		% Styles of quotes in each language
   english=british,
   french=guillemets
]{csquotes}

% All facilities which take a 'cite' argument will not insert
% it directly. They pass it to an auxiliary command called \mkcitation
% which  may be redefined to format the citation.
\renewcommand*{\mkcitation}[1]{{\,}#1}
\renewcommand*{\mkccitation}[1]{ #1}

\SetBlockThreshold{2} % Anzahl von Zeilen

\newenvironment{myquote}%
	{\begin{quote}\small}%
	{\end{quote}}%
\SetBlockEnvironment{myquote}
%\SetCiteCommand{} % Changes citation command


%%% Verweise =============================================================
%
%%% Doc: Documentation inside dtx File
\usepackage[ngerman]{varioref} % Intelligente Querverweise

%%% Listen ===============================================================
%
%
%%% Doc: ftp://tug.ctan.org/pub/tex-archive/macros/latex/contrib/paralist/paralist.pdf
% \usepackage{paralist}
%
%%% Doc: ftp://tug.ctan.org/pub/tex-archive/macros/latex/contrib/enumitem/enumitem.pdf
% Better than 'paralist' and 'enumerate' because it uses a keyvalue interface !
% Do not load together with enumerate.
\IfPackageNotLoaded{enumerate}{
	\usepackage{enumitem}
}
%
%%% Doc: ftp://tug.ctan.org/pub/tex-archive/macros/latex/contrib/ncctools/doc/desclist.pdf
% Improved description environment
%\usepackage{declist}

% ~~~~~~~~~~~~~~~~~~~~~~~~~~~~~~~~~~~~~~~~~~~~~~~~~~~~~~~~~~~~~~~~~~~~~~~~
% figures and placement
% ~~~~~~~~~~~~~~~~~~~~~~~~~~~~~~~~~~~~~~~~~~~~~~~~~~~~~~~~~~~~~~~~~~~~~~~~

%% Bilder und Graphiken ==================================================

%%% Doc: only dtx Package
\usepackage{float}             % Stellt die Option [H] fuer Floats zur Verfgung

%%% Doc: No Documentation
\usepackage{flafter}          % Floats immer erst nach der Referenz setzen

% Defines a \FloatBarrier command, beyond which floats may not
% pass; useful, for example, to ensure all floats for a section
% appear before the next \section command.
\usepackage[
	section		% "\section" command will be redefined with "\FloatBarrier"
]{placeins}
%
%%% Doc: ftp://tug.ctan.org/pub/tex-archive/macros/latex/contrib/subfig/subfig.pdf
% Incompatible: loads package capt-of. Loading of 'capt-of' afterwards will fail therefor
\usepackage{subfig} % Layout wird weiter unten festgelegt !

%%% Bilder von Text Umfliessen lassen : (empfehle wrapfig)
%
%%% Doc: ftp://tug.ctan.org/pub/tex-archive/macros/latex/contrib/wrapfig/wrapfig.sty
\usepackage{wrapfig}	        % defines wrapfigure and wrapfloat
%\setlength{\wrapoverhang}{\marginparwidth} % aeerlapp des Bildes ...
%\addtolength{\wrapoverhang}{\marginparsep} % ... in den margin
\setlength{\intextsep}{0.5\baselineskip} % Platz ober- und unterhalb des Bildes
% \intextsep ignoiert bei draft ???
%\setlength{\columnsep}{1em} % Abstand zum Text

%%% Doc: Documentation inside dtx Package
%\usepackage{floatflt}   	  % LaTeX2e Paket von 1996
                             % [rflt] - Standard float auf der rechten Seite

%%% Doc: ftp://tug.ctan.org/pub/tex-archive/macros/latex209/contrib/picins/picins.doc
%\usepackage{picins}          % LaTeX 2.09 Paket von 1992. aber Layout kombatibel


% Make float placement easier
\renewcommand{\floatpagefraction}{.75} % vorher: .5
\renewcommand{\textfraction}{.1}       % vorher: .2
\renewcommand{\topfraction}{.8}        % vorher: .7
\renewcommand{\bottomfraction}{.5}     % vorher: .3
\setcounter{topnumber}{3}              % vorher: 2
\setcounter{bottomnumber}{2}           % vorher: 1
\setcounter{totalnumber}{5}            % vorher: 3


%%% Doc: ftp://tug.ctan.org/pub/tex-archive/macros/latex/contrib/psfrag/pfgguide.pdf
% \usepackage{psfrag}	% Ersetzen von Zeichen in eps Bildern


%%% Doc: http://www.ctan.org/tex-archive/macros/latex/contrib/sidecap/sidecap.pdf
\usepackage[%
%	outercaption,%	(default) caption is placed always on the outside side
%	innercaption,% caption placed on the inner side
%	leftcaption,%  caption placed on the left side
	rightcaption,% caption placed on the right side
%	wide,%			caption of float my extend into the margin if necessary
%	margincaption,% caption set into margin
	ragged,% caption is set ragged
]{sidecap}

\renewcommand\sidecaptionsep{2em}
%\renewcommand\sidecaptionrelwidth{20}
\sidecaptionvpos{table}{c}
\sidecaptionvpos{figure}{c}


%% Diagramme mit LaTeX ===================================================
%

%%% Doc: ftp://tug.ctan.org/pub/tex-archive/macros/latex/contrib/pict2e/pict2e.pdf
% Neuimplementation der Picture Umgebung.
%
% The new package extends the existing LaTeX picture environment, using
% the familiar technique (cf. the graphics and color packages) of driver
% files.  The package documentation (pict2e.dtx) has a fair number of
% examples of use, showing where things are improved by comparison with
% the LaTeX picture environment.
% \usepackage{pict2e}

%%% Doc: ftp://tug.ctan.org/pub/tex-archive/macros/latex/contrib/curve2e/curve2e.pdf
% Extensions for package pict2e.
%\usepackage{curve2e}
%

% ~~~~~~~~~~~~~~~~~~~~~~~~~~~~~~~~~~~~~~~~~~~~~~~~~~~~~~~~~~~~~~~~~~~~~~~~
% misc packages
% ~~~~~~~~~~~~~~~~~~~~~~~~~~~~~~~~~~~~~~~~~~~~~~~~~~~~~~~~~~~~~~~~~~~~~~~~

\usepackage{makeidx}		% Index
\IfDraft{
  \usepackage{showidx}    % Indizierte Begriffe am Rand (Korrekturlesen)
}

%%% Doc: ftp://tug.ctan.org/pub/tex-archive/macros/latex/contrib/isodate/README
%%% Incompatible: draftcopy
% Tune the output format of dates.
%\usepackage{isodate}

%%% Doc: ftp://tug.ctan.org/pub/tex-archive/macros/latex/contrib/numprint/numprint.pdf
% Modify printing of numbers
%\usepackage{numprint}

%%% Doc: ftp://tug.ctan.org/pub/tex-archive/macros/latex/contrib/nomencl/nomencl.pdf
\usepackage[%
	german,
	%english
]{nomencl}[2005/09/22]

\usepackage[
	%footnote,	% Full names appear in the footnote
	smaller,		% Print acronym in smaller fontsize
	printonlyused %
]{acronym}

%\renewcommand*{\acsfont}[1]{{\smaller\sffamily#1}}


% ~~~~~~~~~~~~~~~~~~~~~~~~~~~~~~~~~~~~~~~~~~~~~~~~~~~~~~~~~~~~~~~~~~~~~~~~
% verbatim packages
% ~~~~~~~~~~~~~~~~~~~~~~~~~~~~~~~~~~~~~~~~~~~~~~~~~~~~~~~~~~~~~~~~~~~~~~~~

%%% Doc: ftp://tug.ctan.org/pub/tex-archive/macros/latex/contrib/upquote/upquote.sty
\usepackage{upquote} % Setzt "richtige" Quotes in verbatim-Umgebung

%%% Doc: No Documentation
% \usepackage{verbatim} %Reimplemntation of the original verbatim

%%% Doc: http://www.cs.brown.edu/system/software/latex/doc/fancyvrb.pdf
% \usepackage{fancyvrb} % Superior Verbatim Class

%% Listings Paket ------------------------------------------------------
%%% Doc: ftp://tug.ctan.org/pub/tex-archive/macros/latex/contrib/listings/listings-1.3.pdf
 \usepackage{listings}

%%% Doc: ftp://tug.ctan.org/pub/tex-archive/macros/latex/contrib/examplep/eurotex_2005_examplep.pdf
% LaTeX Code und Ergebnis nebeneinander darstellen
%\usepackage{examplep}

% ~~~~~~~~~~~~~~~~~~~~~~~~~~~~~~~~~~~~~~~~~~~~~~~~~~~~~~~~~~~~~~~~~~~~~~~~
% science packages
% ~~~~~~~~~~~~~~~~~~~~~~~~~~~~~~~~~~~~~~~~~~~~~~~~~~~~~~~~~~~~~~~~~~~~~~~~

\usepackage{units}

% ~~~~~~~~~~~~~~~~~~~~~~~~~~~~~~~~~~~~~~~~~~~~~~~~~~~~~~~~~~~~~~~~~~~~~~~~
% fancy packages
% ~~~~~~~~~~~~~~~~~~~~~~~~~~~~~~~~~~~~~~~~~~~~~~~~~~~~~~~~~~~~~~~~~~~~~~~~

%%% Doc: No documentation - documented in 'The LaTeX Companion'
% \usepackage{fancybox}   % for shadowbox, ovalbox

%%% Doc: ftp://tug.ctan.org/pub/tex-archive/macros/latex/contrib/misc/framed.sty
% \usepackage{framed}
% \renewcommand\FrameCommand{\fcolorbox{black}{shadecolor}}

\makeatletter
\IfPackageLoaded{framed}{%
   \IfPackageLoaded{marginnote}{%
     \begingroup
         \g@addto@macro\framed{%
            \let\marginnoteleftadjust\FrameSep
                \let\marginnoterightadjust\FrameSep
         }
      \makeatother
  }
}
\makeatother



%%% Doc: No documentation - documented in 'The LaTeX Companion'
% \usepackage{boxedminipage}

%%% Doc: ftp://tug.ctan.org/pub/tex-archive/macros/latex/contrib/lettrine/doc/lettrine.pdf
% Dropping capitals
% \usepackage{lettrine}


% ~~~~~~~~~~~~~~~~~~~~~~~~~~~~~~~~~~~~~~~~~~~~~~~~~~~~~~~~~~~~~~~~~~~~~~~~
% layout packages
% ~~~~~~~~~~~~~~~~~~~~~~~~~~~~~~~~~~~~~~~~~~~~~~~~~~~~~~~~~~~~~~~~~~~~~~~~

%%% Diverse Pakete und Einstellungen =====================================

%%% Doc: Documentation inside dtx file
% Mehere Text-Spalten
\usepackage{multicol}

%\nonfrenchspacing     % liefert extra Platz hinter Satzenden.
                       % Fuer deutschen Text standardmaessig ausgeschaltet!


\usepackage{ellipsis}  % >>Intelligente<< \dots

%% Zeilenabstand =========================================================
%
%%% Doc: ftp://tug.ctan.org/pub/tex-archive/macros/latex/contrib/setspace/setspace.sty
\usepackage{setspace}
%\onehalfspacing		% 1,5-facher Abstand
%\doublespacing		% 2-facher Abstand
% hereafter load 'typearea' again


% Caption fuer nicht fliessende Umgebungen
%%% Doc: ftp://tug.ctan.org/pub/tex-archive/macros/latex/contrib/misc/capt-of.sty
\IfPackageNotLoaded{caption}{
	\usepackage{capt-of} % only load when caption is not loaded. Otherwise compiling will fail.
	%Usage: \captionof{table}[short Titel]{long Titel}
}
%


%%% Doc: ftp://tug.ctan.org/pub/tex-archive/macros/latex/contrib/mcaption/mcaption.pdf
% Captions in Margins
% \usepackage[
% 	top,
% 	bottom
% ]{mcaption}

%%% Example:
% \begin{figure}
%   \begin{margincap}[short caption]{margin caption}
%     \centering
%     \includegraphics{picture}
%   \end{margincap}
% \end{figure}



% \numberwithin{figure}{chapter} %Befehl zum Kapitelweise Nummerieren der Bilder, setzt `amsmath' vorraus
% \numberwithin{table}{chapter}  %Befehl zum Kapitelweise Nummerieren der Tabellen, setzt `amsmath' vorraus

% ~~~~~~~~~~~~~~~~~~~~~~~~~~~~~~~~~~~~~~~~~~~~~~~~~~~~~~~~~~~~~~~~~~~~~~~~
% PDF related packages
% ~~~~~~~~~~~~~~~~~~~~~~~~~~~~~~~~~~~~~~~~~~~~~~~~~~~~~~~~~~~~~~~~~~~~~~~~

%%% Doc: ftp://tug.ctan.org/pub/tex-archive/macros/latex/contrib/microtype/microtype.pdf
% Optischer Randausgleich mit pdfTeX
\ifpdf
\usepackage[%
	expansion=true, % better typography, but with much larger PDF file.
	protrusion=true
]{microtype}
\fi

%%% Doc: ftp://tug.ctan.org/pub/tex-archive/macros/latex/contrib/hyperref/doc/manual.pdf
\usepackage{hyperref}

%%% Doc: ftp://tug.ctan.org/pub/tex-archive/macros/latex/contrib/oberdiek/hypcap.pdf
% Links auf Gleitumgebungen springen nicht zur Beschriftung,
% sondern zum Anfang der Gleitumgebung
\IfPackageLoaded{hyperref}{%
	\usepackage[figure,table]{hypcap}
}

% Auch Abbildung und nicht nur die Nummer wird zum Link (abgeleitet
% aus Posting von Heiko Oberdiek (d09n5p$9md$1@news.BelWue.DE);
% Verwendung: In \abbvref{label} ist ein Beispiel dargestellt
\providecommand*{\figrefname}{Figure~}
\newcommand*{\figref}[1]{%
  \hyperref[fig:#1]{\figrefname{}}\ref{fig:#1}%
}
\providecommand*{\lstrefname}{Listing~}
\newcommand*{\lstref}[1]{%
  \hyperref[lst:#1]{\lstrefname{}}\ref{lst:#1}%
}
% ebenso bei Tabellen
\providecommand*{\tabrefname}{Table~}
\newcommand*{\tabref}[1]{%
  \hyperref[tab:#1]{\tabrefname{}}\ref{tab:#1}%
}
% und Abschnitten
\providecommand*{\secrefname}{Section~}
\newcommand*{\secref}[1]{%
  \hyperref[sec:#1]{\secrefname{}}\ref{sec:#1}%
}
% und Kapiteln
\providecommand*{\chaprefname}{Chapter~}
\newcommand*{\chapref}[1]{%
  \hyperref[ch:#1]{\chaprefname{}}\ref{ch:#1}%
}

%%% Doc: ftp://tug.ctan.org/pub/tex-archive/macros/latex/contrib/pdfpages/pdfpages.pdf
%\usepackage{pdfpages} % Include pages from external PDF documents in LaTeX documents

%%% Doc: ftp://tug.ctan.org/pub/tex-archive/macros/latex/contrib/oberdiek/pdflscape.sty
%\usepackage{pdflscape} %  Querformat mit PDF
%
% Pakete Laden die nach Hyperref geladen werden sollen
\LoadPackagesNow % (ltxtable, tabularx)

% ~~~~~~~~~~~~~~~~~~~~~~~~~~~~~~~~~~~~~~~~~~~~~~~~~~~~~~~~~~~~~~~~~~~~~~~~
% Zus�tzliche Pakete
% ~~~~~~~~~~~~~~~~~~~~~~~~~~~~~~~~~~~~~~~~~~~~~~~~~~~~~~~~~~~~~~~~~~~~~~~~
%%% Doc: ftp://tug.ctan.org/pub/tex-archive/macros/latex/contrib/hyphenat/hyphenat.pdf
% According to documentation the font warnings can be ignored
%\usepackage[htt]{hyphenat} % enable hyphenation of typewriter text word (\textt).

%% Komprimierung von Bildern in PDF ausschalten
% \ifpdf
%    \pdfcompresslevel=0
% \fi
% ~~~~~~~~~~~~~~~~~~~~~~~~~~~~~~~~~~~~~~~~~~~~~~~~~~~~~~~~~~~~~~~~~~~~~~~~
% end of preambel
% ~~~~~~~~~~~~~~~~~~~~~~~~~~~~~~~~~~~~~~~~~~~~~~~~~~~~~~~~~~~~~~~~~~~~~~~~
%\IfPackageLoaded{fancyvrb}{
%	\DefineShortVerb{\|} % Nur mit fancyvrb zusammen laden!
%}


% Auszufuehrende Befehle  ------------------------------------------------
\IfDefined{makeindex}{\makeindex}
\IfDefined{makenomenclature}{\makenomenclature}
%\IfPackageLoaded{minitoc}{\IfElseChapterDefined{\dominitoc}{\dosecttoc}}


\listfiles
%------------------------------------------------------------------------


