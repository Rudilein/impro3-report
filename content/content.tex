% Estmal alles in einem .tex file, evtl später in mehr aufspalten ...

\section{Introduction}
Presslufthammer is an attempt at developing a prototype of the Dremel Query
System \cite{melnik2010dremel} developed by Google. Dremel is purportedly the
technology that powers the Google BigQuery \cite{bigquery} system so we will
mention this here as well and also sometimes refer to the BigQuery online
documentation when talking about features of the Presslufthammer system
and also in comparing the functionality.

Dremel is a system that allows relational queries\footnote{BigQuery
(and Dremel) use a query language that is very similar to SQL: 
\url{https://developers.google.com/bigquery/docs/query-reference}.} to be
performed on a cluster of machines on very large datasets. The focus here
is on aggregation queries that have very small result sets compared to the
processed amounts of data and that are ``interactive'', meaning that the
results are available quick enough to work with the system interactively.

For more in-depth background information about Dremel and BigQuery you should
consult the paper respectively the documentation. We will only highlight
some key components and difficulties in the following sections.

\subsection{Division of Work}

The Dremel systems has a very modular architecture, making it easy to share the
work. We identified to following key components (The person working on that
particular component is mentioned in parentheses):

\begin{itemize}
  \item Columnar data representation (Aljoscha)
  \item Parsing, analysis, transformation and execution of Queries (Aljoscha)
  \item Network architecture (Fabian)
  \item Utilities and stuff (Aljoscha, Fabian)
\end{itemize}

There is one section per bullet point that describes that part of the system.

\subsection{Some Notes}
The Presslufthammer system is implemented in Java and we have also used the
following tools and libraries:

\begin{itemize}
  \item Maven \cite{maven} for the build system
  \item ANTLR \cite{antlr} for implementing the query parser
  \item Netty \cite{netty} for the network layer
  \item slf4j \cite{slf4j} for logging
  \item Google guava libraries \cite{guava} because they are just plain helpful
\end{itemize}

All package names of Presslufthammer are of the form\\
\texttt{de.tuberlin.dima.presslufthammer.*}\\
bu we leave of the first part in the following sections for simplicity and
just refer to package \texttt{data} instead of the full name.


\section{Columnar Data Representation}

\section{Query Processing}

\section{Network Architecture}

\section{Conclusions and Further Work}
